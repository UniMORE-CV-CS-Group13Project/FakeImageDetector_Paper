\section{Note on Denoising Images} \label{App:Denoising}
During our experment phase another preprocessing technique was also applied in the form of image denoising. \par
The denoising process has been achieved using OpenCV as follows; right before moving to the frequency domain:
\begin{codebox}
    \begin{minted}{python3}
denoised_img = cv2.fastNlMeansDenoising(
    src=resized_greyscale_img,
    dst=None,
    h=3.0,
    templateWindowSize=7,
    searchWindowSize=21
)
    \end{minted}
\end{codebox}
This approach should preserve most of the high-frequency noise characteristics typical of image generators. \par
Below are the results coming from a complete re-training of the same network with the denoised image dataset. \par
As shown in the following plots it is noticeable that the class of the Generator 0 is the only one that could benefit from the denoising process. This is probably due to examples coming from the Generator 0 being particularly noisy, and the denoising process helping in isolating significant patterns better. \par
Other classes either get marginal differences or - as is the case for the class of the Generator 2 - a more significant penalty, with the latter loosing in the F1-Score metric from the denoising process, probably due to wiping important information through the additional preprocessing step. \par
All of these observations could still be in the margin of error since their magnitude is in the $0.001\% - 0.01\%$ range.
\begin{figure}[H]
    \centering
    \begin{subfigure}[t]{.45\textwidth}
        \includegraphics[width=\textwidth]{assets/plots/denoise/grouped/loss.pdf}
        \caption{Loss performances over epochs}
        \label{Fig:Loss-denoise}
    \end{subfigure}
\end{figure}
\begin{figure}[H]
    \centering
    \ContinuedFloat
    \begin{subfigure}[t]{.45\textwidth}
        \includegraphics[width=\textwidth]{assets/plots/denoise/grouped/acc.pdf}
        \caption{Accuracy performances over epochs}
        \label{Fig:Acc-denoise}
    \end{subfigure}
    \caption{General metrics performances}
    \label{Fig:GeneralResults-denoise}
\end{figure}
\begin{figure}[H]
    \centering
    \begin{subfigure}[t]{.45\textwidth}
        \includegraphics[width=\textwidth]{assets/plots/denoise/grouped/micro.pdf}
        \caption{Micro performances over epochs}
        \label{Fig:Micro-denoise}
    \end{subfigure}
\end{figure}
\begin{figure}[H]
    \centering
    \ContinuedFloat
    \begin{subfigure}[t]{.45\textwidth}
        \includegraphics[width=\textwidth]{assets/plots/denoise/grouped/macro.pdf}
        \caption{Macro performances over epochs}
        \label{Fig:Macro-denoise}
    \end{subfigure}
\end{figure}
\begin{figure}[H]
    \centering
    \ContinuedFloat
    \begin{subfigure}[t]{.45\textwidth}
        \includegraphics[width=\textwidth]{assets/plots/denoise/grouped/weighted.pdf}
        \caption{Weighted performances over epochs}
        \label{Fig:Weighted-denoise}
    \end{subfigure}
    \caption{Weighted general metrics performances}
    \label{Fig:WeightedResults-denoise}
\end{figure}
\begin{figure}[H]
    \centering
    \begin{subfigure}[t]{.45\textwidth}
        \includegraphics[width=\textwidth]{assets/plots/denoise/grouped/class_0.pdf}
        \caption{Class ``Generator 0'' performances over epochs}
        \label{Fig:Class0-denoise}
    \end{subfigure}
\end{figure}
\begin{figure}[H]
    \centering
    \ContinuedFloat
    \begin{subfigure}[t]{.45\textwidth}
        \includegraphics[width=\textwidth]{assets/plots/denoise/grouped/cm_0.pdf}
        \caption{Class ``Generator 0'' confusion matrices over epochs}
        \label{Fig:Class0cm-denoise}
    \end{subfigure}
\end{figure}
\begin{figure}[H]
    \centering
    \ContinuedFloat
    \begin{subfigure}[t]{.45\textwidth}
        \includegraphics[width=\textwidth]{assets/plots/denoise/grouped/class_1.pdf}
        \caption{Class ``Generator 1'' performances over epochs}
        \label{Fig:Class1-denoise}
    \end{subfigure}
\end{figure}
\begin{figure}[H]
    \centering
    \ContinuedFloat
    \begin{subfigure}[t]{.45\textwidth}
        \includegraphics[width=\textwidth]{assets/plots/denoise/grouped/cm_1.pdf}
        \caption{Class ``Generator 1'' confusion matrices over epochs}
        \label{Fig:Class1cm-denoise}
    \end{subfigure}
\end{figure}
\begin{figure}[H]
    \centering
    \ContinuedFloat
    \begin{subfigure}[t]{.45\textwidth}
        \includegraphics[width=\textwidth]{assets/plots/denoise/grouped/class_2.pdf}
        \caption{Class ``Generator 2'' performances over epochs}
        \label{Fig:Class2-denoise}
    \end{subfigure}
\end{figure}
\begin{figure}[H]
    \centering
    \ContinuedFloat
    \begin{subfigure}[t]{.45\textwidth}
        \includegraphics[width=\textwidth]{assets/plots/denoise/grouped/cm_2.pdf}
        \caption{Class ``Generator 2'' confusion matrices over epochs}
        \label{Fig:Class2cm-denoise}
    \end{subfigure}
\end{figure}
\begin{figure}[H]
    \centering
    \ContinuedFloat
    \begin{subfigure}[t]{.45\textwidth}
        \includegraphics[width=\textwidth]{assets/plots/denoise/grouped/class_3.pdf}
        \caption{Class ``Generator 3'' performances over epochs}
        \label{Fig:Class3-denoise}
    \end{subfigure}
\end{figure}
\begin{figure}[H]
    \centering
    \ContinuedFloat
    \begin{subfigure}[t]{.45\textwidth}
        \includegraphics[width=\textwidth]{assets/plots/denoise/grouped/cm_3.pdf}
        \caption{Class ``Generator 3'' confusion matrices over epochs}
        \label{Fig:Class3cm-denoise}
    \end{subfigure}
\end{figure}
\begin{figure}[H]
    \centering
    \ContinuedFloat
    \begin{subfigure}[t]{.45\textwidth}
        \includegraphics[width=\textwidth]{assets/plots/denoise/grouped/class_4.pdf}
        \caption{Class ``Real Image'' performances over epochs}
        \label{Fig:Class4-denoise}
    \end{subfigure}
\end{figure}
\begin{figure}[H]
    \centering
    \ContinuedFloat
    \begin{subfigure}[t]{.45\textwidth}
        \includegraphics[width=\textwidth]{assets/plots/denoise/grouped/cm_4.pdf}
        \caption{Class ``Real Image'' confusion matrices over epochs}
        \label{Fig:Class4cm-denoise}
    \end{subfigure}
\end{figure}
\begin{figure}[H]
    \centering
    \ContinuedFloat
    \begin{subfigure}[t]{.45\textwidth}
        \includegraphics[width=\textwidth]{assets/plots/denoise/grouped/cm_test.pdf}
        \caption{Confusion matrices in the Test phase}
        \label{Fig:Testcm-denoise}
    \end{subfigure}
    \caption{Per-class metrics performances}
    \label{Fig:PerClassResults-denoise}
\end{figure}